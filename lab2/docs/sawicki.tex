\documentclass[a4paper; 12pt]{article}

% global includes
\usepackage[polish]{babel}
\usepackage[utf8]{inputenc}
\usepackage{polski}
\usepackage{courier} %times, kurier
\usepackage{amsmath}
\usepackage{graphicx}
\usepackage{geometry}
\usepackage{indentfirst}
\usepackage{icomma}
\usepackage{booktabs}

\usepackage{listings}

% local includes
\usepackage[locale=FR]{siunitx}
\sisetup{per-mode=symbol-or-fraction}

\title{Spark - poszukiwanie spójnej składowej}
\author{Jakub Sawicki}
\date{\today}

\begin{document}
\renewcommand{\figurename}{Rys.}
\renewcommand{\tablename}{Tab.}
\renewcommand{\abstractname}{Abstrakt}

\maketitle

\section{Analiza PCAM}

Analiza rozbita została na bloki: partition, communication, agglomeration oraz~mapping.~\cite{foster}

\subsection{Podział}
W~problemie CC podstawowym elementem jest wierzchołek grafu.
Przechowuje on informację o~swoim aktualnym numerze porządkowym.
Odbiera informacje o~numerach porządkowych sąsiadów i~uaktualnia swój, na numer
sąsiada jeżeli ten jest mniejszy niż jego.

\subsection{Komunikacja}
Komunikaty wymieniane są pomiędzy wierzchołkami połączonymi krawędziami.
Rozważany graf jest nieskierowany, w~skierowanym algorytm może dać różne
rezultaty w~zależności od ułożenia kolejności wierzchołków.

W~jednym kroku algorytmu każdy wierzchołek $v$ wysyła $\text{deg}(v)$ komunikatów
do swoich sąsiadów i~tyle też komunikatów odbiera.
W~sumie wymienianych jest więc $2\overline{\text{deg}} * n$ komunikatów,
gdzie $n$ jest ilością wierzchołków.

\subsection{Aglomeracja}
Optymalnie było by podzielić graf na pewną liczbę składowych, które są ze sobą
dość gęsto połączone.
Dzięki temu większość komunikacji odbywała by się wewnątrz danej składowej.
Pomiędzy nimi powinno być relatywnie niewiele połączeń.
Nie jest jednak łatwo przeprowadzić takiej analizy na dużych grafach.

Nieoptymalny podział może nastąpić poprzez podział wierzchołków na $m$ grup na
podstawie funkcji hashującej.
Wtedy jeśli średni stopień wierzchołka w~grafie wynosi $\overline{\text{deg}}$
to mamy $s_{\text{part}} = \frac{n}{m} \overline{\text{deg}}$ komunikatów
wymienianych przez każdą część.
Z~tego $\frac{m-1}{m} s_{\text{part}}$ wymieniane jest z~innymi składowymi.

\subsection{Mapowanie}
Posiadając informacje o~ilości połączeń między poszczególnymi składowymi
możliwe jest takie rozmieszczenie ich na węzłach obliczeniowych aby
zminimalizować potrzebną komunikację.

\section{Implementacja}

\section{Wydajność}

\begin{thebibliography}{9}
    \bibitem{foster}
        I.~Foster: \emph{Designing and Building Parallel Programs}
        \texttt{http://www.mcs.anl.gov/dbpp/} dostęp 2015.10.14
\end{thebibliography}

\end{document}
