\documentclass[a4paper; 12pt]{article}

% global includes
\usepackage[polish]{babel}
\usepackage[utf8]{inputenc}
\usepackage{polski}
\usepackage{courier} %times, kurier
\usepackage{amsmath}
\usepackage{graphicx}
\usepackage{geometry}
\usepackage{indentfirst}
\usepackage{icomma}
\usepackage{booktabs}
\usepackage{hyperref}
\usepackage{pgfplots}
\pgfplotsset{
    compat=1.12,
    width=9cm,
    %yticklabel={\num[round-mode=figures,round-precision=2]{\tick}},
    y axis style/.style={
        yticklabel style=#1,
        ylabel style=#1,
        y axis line style=#1,
        ytick style=#1
    }
}

\usepackage{listings}


% local includes
\usepackage[locale=FR]{siunitx}
\sisetup{per-mode=symbol-or-fraction}

\title{Bitonic merge-sort}
\author{Jakub Sawicki}
\date{\today}

\begin{document}
\renewcommand{\figurename}{Rys.}
\renewcommand{\tablename}{Tab.}
\renewcommand{\abstractname}{Abstrakt}

\maketitle

\section{Algorytm}

Jako podstawę pod algorytm wykorzystałem opis i~kod bitonicznego merge-sorta
z~\cite{parallel}.
Został on zmodyfikowany tak, aby przyjmował jako wejście rekordy wygenerowane
przez \texttt{gensort} i~zapisywał je w~plikach weryfikowalnych przez
\texttt{valsort}~\cite{sortalgo}.

Algorytm ten oparty jest na architekturze hiper-kostki.
Musi być uruchomiony na ilości procesorów równej potędze dwójki.
Każdy procesor wykonuje merge-sort na swojej części danych.
Następnie procesory wymieniają się danymi ze swoimi partnerami sortując
powstające kawałki.
Ta część wymaga $\frac{\log_2^2(n_p)}{2}$ iteracji, gdzie $n_p$ to ilość
dostępnych procesorów.


\section{Wyniki}

Testy wydajnościowe zostały wykonane dla czterech wielkości danych wejściowych:
640MB, 1280MB, 2560MB oraz 5120MB.
Co odpowiada odpowiednio 6,4M, 12,8M, 25,6M oraz 51,2M rekordów (każdy po 100
bajtów).

Nie udało się uruchomić algorytmu dla dużych problemów i~niewielkiej ilości węzłów.
Wynikało to z~użycia typu int, do określania wielkości alokowanego bufora,
powodując błąd jeśli był on większy niż około 2GB.

Wyniki pokazane są na Rys.~\ref{fig:results1}.
Nie ma przyspieszenia, jeżeli obliczenia wykonywane są na pojedynczym węźle, do
8 rdzeni.
Dla większej ich ilości wydajność zaczyna wzrastać.

\begin{figure}[p]
    \centering
    \includegraphics[width=.5\textwidth]{runs.png}
    \includegraphics[width=.5\textwidth]{speedup.png}
    \includegraphics[width=.5\textwidth]{rate.png}
    \caption{Wykresy czasu wykonania, przyspieszenia oraz tempa przetwarzania
        danych dla różnych wielkości danych wejściowych.
        Dane wejściowe miały wielkości 640MB, 1280MB, 2560MB oraz 5210MB.}
    \label{fig:results1}
\end{figure}



\begin{thebibliography}{9}
    \bibitem{parallel}
        \emph{Divide-and-Conquer Parallelization Paradigm}
        \url{http://cacs.usc.edu/education/cs653/02-3DC.pdf} dostęp 2016.01.14
    \bibitem{sortalgo}
        \emph{Sort algorithms benchmark}
        \url{http://sortbenchmark.org/} dostęp 2016.01.14
\end{thebibliography}

\end{document}
